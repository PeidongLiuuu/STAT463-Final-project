% Options for packages loaded elsewhere
\PassOptionsToPackage{unicode}{hyperref}
\PassOptionsToPackage{hyphens}{url}
%
\documentclass[
]{article}
\usepackage{amsmath,amssymb}
\usepackage{iftex}
\ifPDFTeX
  \usepackage[T1]{fontenc}
  \usepackage[utf8]{inputenc}
  \usepackage{textcomp} % provide euro and other symbols
\else % if luatex or xetex
  \usepackage{unicode-math} % this also loads fontspec
  \defaultfontfeatures{Scale=MatchLowercase}
  \defaultfontfeatures[\rmfamily]{Ligatures=TeX,Scale=1}
\fi
\usepackage{lmodern}
\ifPDFTeX\else
  % xetex/luatex font selection
\fi
% Use upquote if available, for straight quotes in verbatim environments
\IfFileExists{upquote.sty}{\usepackage{upquote}}{}
\IfFileExists{microtype.sty}{% use microtype if available
  \usepackage[]{microtype}
  \UseMicrotypeSet[protrusion]{basicmath} % disable protrusion for tt fonts
}{}
\makeatletter
\@ifundefined{KOMAClassName}{% if non-KOMA class
  \IfFileExists{parskip.sty}{%
    \usepackage{parskip}
  }{% else
    \setlength{\parindent}{0pt}
    \setlength{\parskip}{6pt plus 2pt minus 1pt}}
}{% if KOMA class
  \KOMAoptions{parskip=half}}
\makeatother
\usepackage{xcolor}
\usepackage[margin=1in]{geometry}
\usepackage{color}
\usepackage{fancyvrb}
\newcommand{\VerbBar}{|}
\newcommand{\VERB}{\Verb[commandchars=\\\{\}]}
\DefineVerbatimEnvironment{Highlighting}{Verbatim}{commandchars=\\\{\}}
% Add ',fontsize=\small' for more characters per line
\usepackage{framed}
\definecolor{shadecolor}{RGB}{248,248,248}
\newenvironment{Shaded}{\begin{snugshade}}{\end{snugshade}}
\newcommand{\AlertTok}[1]{\textcolor[rgb]{0.94,0.16,0.16}{#1}}
\newcommand{\AnnotationTok}[1]{\textcolor[rgb]{0.56,0.35,0.01}{\textbf{\textit{#1}}}}
\newcommand{\AttributeTok}[1]{\textcolor[rgb]{0.13,0.29,0.53}{#1}}
\newcommand{\BaseNTok}[1]{\textcolor[rgb]{0.00,0.00,0.81}{#1}}
\newcommand{\BuiltInTok}[1]{#1}
\newcommand{\CharTok}[1]{\textcolor[rgb]{0.31,0.60,0.02}{#1}}
\newcommand{\CommentTok}[1]{\textcolor[rgb]{0.56,0.35,0.01}{\textit{#1}}}
\newcommand{\CommentVarTok}[1]{\textcolor[rgb]{0.56,0.35,0.01}{\textbf{\textit{#1}}}}
\newcommand{\ConstantTok}[1]{\textcolor[rgb]{0.56,0.35,0.01}{#1}}
\newcommand{\ControlFlowTok}[1]{\textcolor[rgb]{0.13,0.29,0.53}{\textbf{#1}}}
\newcommand{\DataTypeTok}[1]{\textcolor[rgb]{0.13,0.29,0.53}{#1}}
\newcommand{\DecValTok}[1]{\textcolor[rgb]{0.00,0.00,0.81}{#1}}
\newcommand{\DocumentationTok}[1]{\textcolor[rgb]{0.56,0.35,0.01}{\textbf{\textit{#1}}}}
\newcommand{\ErrorTok}[1]{\textcolor[rgb]{0.64,0.00,0.00}{\textbf{#1}}}
\newcommand{\ExtensionTok}[1]{#1}
\newcommand{\FloatTok}[1]{\textcolor[rgb]{0.00,0.00,0.81}{#1}}
\newcommand{\FunctionTok}[1]{\textcolor[rgb]{0.13,0.29,0.53}{\textbf{#1}}}
\newcommand{\ImportTok}[1]{#1}
\newcommand{\InformationTok}[1]{\textcolor[rgb]{0.56,0.35,0.01}{\textbf{\textit{#1}}}}
\newcommand{\KeywordTok}[1]{\textcolor[rgb]{0.13,0.29,0.53}{\textbf{#1}}}
\newcommand{\NormalTok}[1]{#1}
\newcommand{\OperatorTok}[1]{\textcolor[rgb]{0.81,0.36,0.00}{\textbf{#1}}}
\newcommand{\OtherTok}[1]{\textcolor[rgb]{0.56,0.35,0.01}{#1}}
\newcommand{\PreprocessorTok}[1]{\textcolor[rgb]{0.56,0.35,0.01}{\textit{#1}}}
\newcommand{\RegionMarkerTok}[1]{#1}
\newcommand{\SpecialCharTok}[1]{\textcolor[rgb]{0.81,0.36,0.00}{\textbf{#1}}}
\newcommand{\SpecialStringTok}[1]{\textcolor[rgb]{0.31,0.60,0.02}{#1}}
\newcommand{\StringTok}[1]{\textcolor[rgb]{0.31,0.60,0.02}{#1}}
\newcommand{\VariableTok}[1]{\textcolor[rgb]{0.00,0.00,0.00}{#1}}
\newcommand{\VerbatimStringTok}[1]{\textcolor[rgb]{0.31,0.60,0.02}{#1}}
\newcommand{\WarningTok}[1]{\textcolor[rgb]{0.56,0.35,0.01}{\textbf{\textit{#1}}}}
\usepackage{graphicx}
\makeatletter
\def\maxwidth{\ifdim\Gin@nat@width>\linewidth\linewidth\else\Gin@nat@width\fi}
\def\maxheight{\ifdim\Gin@nat@height>\textheight\textheight\else\Gin@nat@height\fi}
\makeatother
% Scale images if necessary, so that they will not overflow the page
% margins by default, and it is still possible to overwrite the defaults
% using explicit options in \includegraphics[width, height, ...]{}
\setkeys{Gin}{width=\maxwidth,height=\maxheight,keepaspectratio}
% Set default figure placement to htbp
\makeatletter
\def\fps@figure{htbp}
\makeatother
\setlength{\emergencystretch}{3em} % prevent overfull lines
\providecommand{\tightlist}{%
  \setlength{\itemsep}{0pt}\setlength{\parskip}{0pt}}
\setcounter{secnumdepth}{-\maxdimen} % remove section numbering
\newlength{\cslhangindent}
\setlength{\cslhangindent}{1.5em}
\newlength{\csllabelwidth}
\setlength{\csllabelwidth}{3em}
\newlength{\cslentryspacingunit} % times entry-spacing
\setlength{\cslentryspacingunit}{\parskip}
\newenvironment{CSLReferences}[2] % #1 hanging-ident, #2 entry spacing
 {% don't indent paragraphs
  \setlength{\parindent}{0pt}
  % turn on hanging indent if param 1 is 1
  \ifodd #1
  \let\oldpar\par
  \def\par{\hangindent=\cslhangindent\oldpar}
  \fi
  % set entry spacing
  \setlength{\parskip}{#2\cslentryspacingunit}
 }%
 {}
\usepackage{calc}
\newcommand{\CSLBlock}[1]{#1\hfill\break}
\newcommand{\CSLLeftMargin}[1]{\parbox[t]{\csllabelwidth}{#1}}
\newcommand{\CSLRightInline}[1]{\parbox[t]{\linewidth - \csllabelwidth}{#1}\break}
\newcommand{\CSLIndent}[1]{\hspace{\cslhangindent}#1}
\ifLuaTeX
  \usepackage{selnolig}  % disable illegal ligatures
\fi
\IfFileExists{bookmark.sty}{\usepackage{bookmark}}{\usepackage{hyperref}}
\IfFileExists{xurl.sty}{\usepackage{xurl}}{} % add URL line breaks if available
\urlstyle{same}
\hypersetup{
  pdftitle={Stat 463 Final Project},
  pdfauthor={Gustav Vu},
  hidelinks,
  pdfcreator={LaTeX via pandoc}}

\title{Stat 463 Final Project}
\author{Gustav Vu}
\date{2024-11-12}

\begin{document}
\maketitle

Load Libraries

Load Data

\begin{Shaded}
\begin{Highlighting}[]
\CommentTok{\# Fetch the data}
\NormalTok{co2Data }\OtherTok{\textless{}{-}} \FunctionTok{read.csv}\NormalTok{(}\StringTok{"https://ourworldindata.org/grapher/annual{-}co2{-}emissions{-}per{-}country.csv?v=1\&csvType=filtered\&useColumnShortNames=false\&country=\textasciitilde{}OWID\_WRL"}\NormalTok{)}


\NormalTok{YearlyTemp }\OtherTok{\textless{}{-}} \FunctionTok{read.csv}\NormalTok{(}\StringTok{"Yearly.csv"}\NormalTok{)}
\end{Highlighting}
\end{Shaded}

\hypertarget{clean-data}{%
\section{Clean Data}\label{clean-data}}

\begin{Shaded}
\begin{Highlighting}[]
\NormalTok{YearlyTemp }\OtherTok{\textless{}{-}}\NormalTok{ YearlyTemp[YearlyTemp}\SpecialCharTok{$}\NormalTok{Source }\SpecialCharTok{==} \StringTok{"gcag"}\NormalTok{, ] }\CommentTok{\# keep source constant.}
\NormalTok{c02\_TS }\OtherTok{\textless{}{-}}\NormalTok{ co2Data[, }\FunctionTok{c}\NormalTok{(}\StringTok{"Year"}\NormalTok{, }\StringTok{"Annual.CO..emissions"}\NormalTok{)]}
\NormalTok{c02\_TS }\OtherTok{\textless{}{-}}\NormalTok{ c02\_TS[}\FunctionTok{order}\NormalTok{(c02\_TS}\SpecialCharTok{$}\NormalTok{Year), ]}
\NormalTok{c02\_TS }\OtherTok{\textless{}{-}} \FunctionTok{ts}\NormalTok{(c02\_TS}\SpecialCharTok{$}\NormalTok{Annual.CO..emissions, }
                   \AttributeTok{start =} \FunctionTok{min}\NormalTok{(c02\_TS}\SpecialCharTok{$}\NormalTok{Year), }
                   \AttributeTok{end =} \FunctionTok{max}\NormalTok{(c02\_TS}\SpecialCharTok{$}\NormalTok{Year), }
                   \AttributeTok{frequency =} \DecValTok{1}\NormalTok{)}

\NormalTok{temp\_TS }\OtherTok{\textless{}{-}}\NormalTok{ YearlyTemp[, }\FunctionTok{c}\NormalTok{(}\StringTok{"Year"}\NormalTok{, }\StringTok{"Mean"}\NormalTok{)]}
\NormalTok{temp\_TS }\OtherTok{\textless{}{-}}\NormalTok{ temp\_TS[}\FunctionTok{order}\NormalTok{(temp\_TS}\SpecialCharTok{$}\NormalTok{Year), ]}
\NormalTok{temp\_TS }\OtherTok{\textless{}{-}} \FunctionTok{ts}\NormalTok{(temp\_TS}\SpecialCharTok{$}\NormalTok{Mean, }
                   \AttributeTok{start =} \FunctionTok{min}\NormalTok{(temp\_TS}\SpecialCharTok{$}\NormalTok{Year), }
                   \AttributeTok{end =} \FunctionTok{max}\NormalTok{(temp\_TS}\SpecialCharTok{$}\NormalTok{Year), }
                   \AttributeTok{frequency =} \DecValTok{1}\NormalTok{)}

\NormalTok{dat }\OtherTok{\textless{}{-}} \FunctionTok{ts.intersect}\NormalTok{(c02\_TS, temp\_TS)}
\end{Highlighting}
\end{Shaded}

\hypertarget{introduction}{%
\section{Introduction}\label{introduction}}

For decades, the scientific community, policymakers, and global
organizations have been grappling with the pressing issue of global
warming, aiming to identify, develop, and implement the most effective
strategies to mitigate its effects. One of the most critical steps in
addressing this challenge is to thoroughly investigate its underlying
causes, particularly the factors contributing to the increased
concentration of carbon dioxide (\(co_2\)) in the atmosphere. As a
primary greenhouse gas, \(co_2\) plays a pivotal role in the
intensification of the greenhouse effect, which has directly and
indirectly led to a steady rise in global average temperatures.
Understanding the historical trends and sources of \(co_2\) emissions is
crucial to comprehending the broader dynamics of climate change.

This research specifically focuses on analyzing the significant boom in
\(co_2\) emissions that occurred during the Second Industrial
Revolution, a period marked by rapid industrialization, technological
advancements, and fossil fuel exploitation. By evaluating the
socio-economic activities and technological developments of this
transformative era, we aim to uncover the extent to which these factors
contributed to the expansion of greenhouse gas levels and how they set
the stage for the modern challenges of global warming. Through this
analysis, we hope to provide insights into historical emission trends
and their implications for current and future efforts to combat climate
change.

\hypertarget{background}{%
\section{Background}\label{background}}

\hypertarget{global-warming-common-issue}{%
\subsubsection{Global warming common
issue}\label{global-warming-common-issue}}

The Intergovernmental Panel on Climate Change (IPCC) issued a report in
1990 utilizing global mean near surface temperature which raised concern
on the rate of increasing. {[}5{]}. This concern raised the question of
further research on greenhouse effect of the

Anthropological contribution on the emission of carbon dioxide that
mainly draws from the effect of human activity {[}3{]} on a large scale
acts as the main motive of the extreme spur especially after the second
industrial revolution era.{[}2{]} Predominantly starts from western
europe and north america where both regions exceeded 10 ton per capita
as from the region report (Ritchie, Rosado, and Roser 2023). A
combination of boost in different industries which is facilitated by the
second industrial revolution including chemical and transportation
domain {[}4{]} after the mid 1880s followed by the expansion boom leads
to the official start of boosting the aggregate global carbon dioxide
emission. This major event placed a major factor on the \(co_2\)
emission which reflected on the overall data trend in terms of human
intervention.

\hypertarget{method}{%
\section{Method}\label{method}}

In this study, we determine the effect of \(co_2\) on temperature and
compare the trend found through constructing a time series model for
forecasting and the Arch-Garch method for taking care of the variance
change aspect as time progresses. We use linear regression and
Auto-Aggressive-Moving-Average(ARIMA) models for stationary and
stochastic respectively. The model takes in consideration the data
behavior change due to real world events.

\begin{Shaded}
\begin{Highlighting}[]
\FunctionTok{plot}\NormalTok{(dat[,}\DecValTok{1}\NormalTok{])}
\end{Highlighting}
\end{Shaded}

\includegraphics{Final-Project_files/figure-latex/unnamed-chunk-4-1.pdf}

\begin{Shaded}
\begin{Highlighting}[]
\FunctionTok{plot}\NormalTok{(dat[,}\DecValTok{2}\NormalTok{])}
\end{Highlighting}
\end{Shaded}

\includegraphics{Final-Project_files/figure-latex/unnamed-chunk-4-2.pdf}

\begin{Shaded}
\begin{Highlighting}[]
\FunctionTok{acf2}\NormalTok{(}\FunctionTok{diff}\NormalTok{(}\FunctionTok{log}\NormalTok{(dat[,}\DecValTok{1}\NormalTok{])))}
\end{Highlighting}
\end{Shaded}

\includegraphics{Final-Project_files/figure-latex/unnamed-chunk-4-3.pdf}

\begin{verbatim}
##       [,1]  [,2] [,3] [,4]  [,5] [,6] [,7]  [,8]  [,9] [,10] [,11] [,12] [,13]
## ACF  -0.02 -0.04 0.04 0.11 -0.08    0 0.26 -0.05 -0.07  0.06  0.14 -0.01  0.22
## PACF -0.02 -0.04 0.04 0.11 -0.07    0 0.25 -0.05 -0.04  0.04  0.09  0.04  0.25
##      [,14] [,15] [,16] [,17] [,18] [,19] [,20] [,21] [,22] [,23] [,24]
## ACF    0.0 -0.03  0.02  0.18 -0.06  0.05 -0.02 -0.03 -0.08  0.05  0.08
## PACF  -0.1 -0.01  0.06  0.13 -0.10  0.09 -0.18 -0.01 -0.04  0.01 -0.06
\end{verbatim}

\hypertarget{model-deterministic-part.}{%
\section{Model Deterministic Part.}\label{model-deterministic-part.}}

Looking at our plot It seems as if events occurred around 1950, and 1900
start of second industrial revolution and end of World War II.

\begin{Shaded}
\begin{Highlighting}[]
\FunctionTok{plot}\NormalTok{(dat[,}\DecValTok{2}\NormalTok{])}
\end{Highlighting}
\end{Shaded}

\includegraphics{Final-Project_files/figure-latex/unnamed-chunk-5-1.pdf}

\begin{Shaded}
\begin{Highlighting}[]
\NormalTok{Afterww2 }\OtherTok{\textless{}{-}} \FunctionTok{as.numeric}\NormalTok{(}\FunctionTok{time}\NormalTok{(dat)}\SpecialCharTok{\textgreater{}=} \DecValTok{1945}\NormalTok{)}
\NormalTok{IndustryRev }\OtherTok{\textless{}{-}} \FunctionTok{as.numeric}\NormalTok{(}\FunctionTok{time}\NormalTok{(dat)}\SpecialCharTok{\textgreater{}=} \DecValTok{1900}\NormalTok{)}
\NormalTok{I1976 }\OtherTok{\textless{}{-}} \FunctionTok{as.numeric}\NormalTok{(}\FunctionTok{time}\NormalTok{(dat)}\SpecialCharTok{\textgreater{}=} \DecValTok{1976}\NormalTok{)}

\NormalTok{model2 }\OtherTok{\textless{}{-}} \FunctionTok{lm}\NormalTok{(dat[,}\DecValTok{2}\NormalTok{] }\SpecialCharTok{\textasciitilde{}} \FunctionTok{time}\NormalTok{(dat)}\SpecialCharTok{+}\FunctionTok{time}\NormalTok{(dat)}\SpecialCharTok{*}\NormalTok{Afterww2}\SpecialCharTok{+} \FunctionTok{time}\NormalTok{(dat)}\SpecialCharTok{*}\NormalTok{IndustryRev, }\AttributeTok{data =}\NormalTok{ dat)}
\CommentTok{\#plot(model2)}
\FunctionTok{par}\NormalTok{(}\AttributeTok{mfrow=}\FunctionTok{c}\NormalTok{(}\DecValTok{2}\NormalTok{,}\DecValTok{2}\NormalTok{))}
\FunctionTok{plot}\NormalTok{(model2, }\AttributeTok{whixh =} \DecValTok{1}\SpecialCharTok{:}\DecValTok{4}\NormalTok{)}
\end{Highlighting}
\end{Shaded}

\begin{verbatim}
## Warning in plot.window(...): "whixh" is not a graphical parameter
\end{verbatim}

\begin{verbatim}
## Warning in plot.xy(xy, type, ...): "whixh" is not a graphical parameter
\end{verbatim}

\begin{verbatim}
## Warning in axis(side = side, at = at, labels = labels, ...): "whixh" is not a
## graphical parameter

## Warning in axis(side = side, at = at, labels = labels, ...): "whixh" is not a
## graphical parameter
\end{verbatim}

\begin{verbatim}
## Warning in box(...): "whixh" is not a graphical parameter
\end{verbatim}

\begin{verbatim}
## Warning in title(...): "whixh" is not a graphical parameter
\end{verbatim}

\begin{verbatim}
## Warning in plot.xy(xy.coords(x, y), type = type, ...): "whixh" is not a
## graphical parameter
\end{verbatim}

\begin{verbatim}
## Warning in plot.window(...): "whixh" is not a graphical parameter
\end{verbatim}

\begin{verbatim}
## Warning in plot.xy(xy, type, ...): "whixh" is not a graphical parameter
\end{verbatim}

\begin{verbatim}
## Warning in axis(side = side, at = at, labels = labels, ...): "whixh" is not a
## graphical parameter

## Warning in axis(side = side, at = at, labels = labels, ...): "whixh" is not a
## graphical parameter
\end{verbatim}

\begin{verbatim}
## Warning in box(...): "whixh" is not a graphical parameter
\end{verbatim}

\begin{verbatim}
## Warning in title(...): "whixh" is not a graphical parameter
\end{verbatim}

\begin{verbatim}
## Warning in plot.window(...): "whixh" is not a graphical parameter
\end{verbatim}

\begin{verbatim}
## Warning in plot.xy(xy, type, ...): "whixh" is not a graphical parameter
\end{verbatim}

\begin{verbatim}
## Warning in axis(side = side, at = at, labels = labels, ...): "whixh" is not a
## graphical parameter

## Warning in axis(side = side, at = at, labels = labels, ...): "whixh" is not a
## graphical parameter
\end{verbatim}

\begin{verbatim}
## Warning in box(...): "whixh" is not a graphical parameter
\end{verbatim}

\begin{verbatim}
## Warning in title(...): "whixh" is not a graphical parameter
\end{verbatim}

\begin{verbatim}
## Warning in plot.xy(xy.coords(x, y), type = type, ...): "whixh" is not a
## graphical parameter
\end{verbatim}

\begin{verbatim}
## Warning in plot.window(...): "whixh" is not a graphical parameter
\end{verbatim}

\begin{verbatim}
## Warning in plot.xy(xy, type, ...): "whixh" is not a graphical parameter
\end{verbatim}

\begin{verbatim}
## Warning in axis(side = side, at = at, labels = labels, ...): "whixh" is not a
## graphical parameter

## Warning in axis(side = side, at = at, labels = labels, ...): "whixh" is not a
## graphical parameter
\end{verbatim}

\begin{verbatim}
## Warning in box(...): "whixh" is not a graphical parameter
\end{verbatim}

\begin{verbatim}
## Warning in title(...): "whixh" is not a graphical parameter
\end{verbatim}

\begin{verbatim}
## Warning in plot.xy(xy.coords(x, y), type = type, ...): "whixh" is not a
## graphical parameter
\end{verbatim}

\includegraphics{Final-Project_files/figure-latex/unnamed-chunk-5-2.pdf}

\begin{Shaded}
\begin{Highlighting}[]
\FunctionTok{summary}\NormalTok{(model2)}
\end{Highlighting}
\end{Shaded}

\begin{verbatim}
## 
## Call:
## lm(formula = dat[, 2] ~ time(dat) + time(dat) * Afterww2 + time(dat) * 
##     IndustryRev, data = dat)
## 
## Residuals:
##      Min       1Q   Median       3Q      Max 
## -0.32823 -0.09304  0.00055  0.06970  0.36771 
## 
## Coefficients:
##                         Estimate Std. Error t value Pr(>|t|)    
## (Intercept)             2.551782   2.306113   1.107    0.270    
## time(dat)              -0.001553   0.001230  -1.262    0.209    
## Afterww2               -4.543077   3.029908  -1.499    0.136    
## IndustryRev           -25.753919   3.603931  -7.146 2.60e-11 ***
## time(dat):Afterww2      0.002170   0.001568   1.383    0.168    
## time(dat):IndustryRev   0.013481   0.001895   7.115 3.08e-11 ***
## ---
## Signif. codes:  0 '***' 0.001 '**' 0.01 '*' 0.05 '.' 0.1 ' ' 1
## 
## Residual standard error: 0.1255 on 168 degrees of freedom
## Multiple R-squared:  0.8936, Adjusted R-squared:  0.8904 
## F-statistic: 282.1 on 5 and 168 DF,  p-value: < 2.2e-16
\end{verbatim}

\begin{Shaded}
\begin{Highlighting}[]
\FunctionTok{plot}\NormalTok{(}\FunctionTok{resid}\NormalTok{(model2))}

\FunctionTok{acf}\NormalTok{(}\FunctionTok{resid}\NormalTok{(model2))}
\FunctionTok{pacf}\NormalTok{(}\FunctionTok{resid}\NormalTok{(model2))}
\end{Highlighting}
\end{Shaded}

\includegraphics{Final-Project_files/figure-latex/unnamed-chunk-5-3.pdf}

ACF tails off or cuts off after lag 2. PACF cuts off after lag 1. I Will
fit an arima(1,0,0) and arima(2,0,2) to see which fits better.

\begin{Shaded}
\begin{Highlighting}[]
\NormalTok{model.matrix }\OtherTok{=} \FunctionTok{model.matrix}\NormalTok{(}\AttributeTok{object=} \SpecialCharTok{\textasciitilde{}} \FunctionTok{time}\NormalTok{(dat)}\SpecialCharTok{+}\FunctionTok{time}\NormalTok{(dat)}\SpecialCharTok{*}\NormalTok{Afterww2}\SpecialCharTok{+} \FunctionTok{time}\NormalTok{(dat)}\SpecialCharTok{*}\NormalTok{IndustryRev}\DecValTok{{-}1}\NormalTok{)}

\NormalTok{model.ts }\OtherTok{\textless{}{-}} \FunctionTok{arima}\NormalTok{(}\AttributeTok{x =}\NormalTok{ dat[,}\DecValTok{2}\NormalTok{], }\AttributeTok{order =} \FunctionTok{c}\NormalTok{(}\DecValTok{1}\NormalTok{,}\DecValTok{0}\NormalTok{,}\DecValTok{0}\NormalTok{),  }\AttributeTok{xreg =}\NormalTok{ model.matrix, }\AttributeTok{method =} \StringTok{"ML"}\NormalTok{)}

\FunctionTok{tsdiag}\NormalTok{(model.ts)}
\end{Highlighting}
\end{Shaded}

\includegraphics{Final-Project_files/figure-latex/unnamed-chunk-6-1.pdf}

\begin{Shaded}
\begin{Highlighting}[]
\FunctionTok{acf}\NormalTok{(}\FunctionTok{resid}\NormalTok{(model.ts))}
\end{Highlighting}
\end{Shaded}

\includegraphics{Final-Project_files/figure-latex/unnamed-chunk-6-2.pdf}

\begin{Shaded}
\begin{Highlighting}[]
\FunctionTok{pacf}\NormalTok{(}\FunctionTok{resid}\NormalTok{(model.ts))}
\end{Highlighting}
\end{Shaded}

\includegraphics{Final-Project_files/figure-latex/unnamed-chunk-6-3.pdf}

\begin{Shaded}
\begin{Highlighting}[]
\FunctionTok{coeftest}\NormalTok{(model.ts)}
\end{Highlighting}
\end{Shaded}

\begin{verbatim}
## 
## z test of coefficients:
## 
##                          Estimate  Std. Error z value  Pr(>|z|)    
## ar1                     0.5561333   0.0801484  6.9388 3.955e-12 ***
## intercept               4.0105058   3.9954799  1.0038 0.3154940    
## time(dat)              -0.0023375   0.0021384 -1.0931 0.2743316    
## Afterww2               -9.4970649   5.4416527 -1.7453 0.0809407 .  
## IndustryRev           -22.2203007   6.4850219 -3.4264 0.0006116 ***
## time(dat):Afterww2      0.0047519   0.0028103  1.6909 0.0908609 .  
## time(dat):IndustryRev   0.0116694   0.0034080  3.4241 0.0006168 ***
## ---
## Signif. codes:  0 '***' 0.001 '**' 0.01 '*' 0.05 '.' 0.1 ' ' 1
\end{verbatim}

\begin{Shaded}
\begin{Highlighting}[]
\CommentTok{\#model.matrix = model.matrix(object= \textasciitilde{} time(tmp)+time(tmp)*Afterww2+ time(tmp)*IndustryRev{-}1)}




\NormalTok{model.matrix }\OtherTok{=} \FunctionTok{model.matrix}\NormalTok{(}\AttributeTok{object=} \SpecialCharTok{\textasciitilde{}} \FunctionTok{time}\NormalTok{(dat)}\SpecialCharTok{+}\FunctionTok{time}\NormalTok{(dat)}\SpecialCharTok{*}\NormalTok{Afterww2}\SpecialCharTok{+} \FunctionTok{time}\NormalTok{(dat)}\SpecialCharTok{*}\NormalTok{IndustryRev }\SpecialCharTok{+}\NormalTok{ I1976}\SpecialCharTok{*}\FunctionTok{time}\NormalTok{(dat) }\SpecialCharTok{{-}}\DecValTok{1}\NormalTok{)}

\NormalTok{model.ts }\OtherTok{\textless{}{-}} \FunctionTok{arima}\NormalTok{(}\AttributeTok{x =}\NormalTok{ dat[,}\DecValTok{2}\NormalTok{], }\AttributeTok{order =} \FunctionTok{c}\NormalTok{(}\DecValTok{0}\NormalTok{,}\DecValTok{0}\NormalTok{,}\DecValTok{1}\NormalTok{),  }\AttributeTok{xreg =}\NormalTok{ model.matrix, }\AttributeTok{method =} \StringTok{"ML"}\NormalTok{)}

\FunctionTok{tsdiag}\NormalTok{(model.ts)}
\end{Highlighting}
\end{Shaded}

\includegraphics{Final-Project_files/figure-latex/unnamed-chunk-7-1.pdf}

\begin{Shaded}
\begin{Highlighting}[]
\FunctionTok{acf}\NormalTok{(}\FunctionTok{resid}\NormalTok{(model.ts))}
\end{Highlighting}
\end{Shaded}

\includegraphics{Final-Project_files/figure-latex/unnamed-chunk-7-2.pdf}

\begin{Shaded}
\begin{Highlighting}[]
\FunctionTok{pacf}\NormalTok{(}\FunctionTok{resid}\NormalTok{(model.ts))}
\end{Highlighting}
\end{Shaded}

\includegraphics{Final-Project_files/figure-latex/unnamed-chunk-7-3.pdf}

\begin{Shaded}
\begin{Highlighting}[]
\FunctionTok{coeftest}\NormalTok{(model.ts)}
\end{Highlighting}
\end{Shaded}

\begin{verbatim}
## 
## z test of coefficients:
## 
##                          Estimate  Std. Error z value  Pr(>|z|)    
## ma1                     0.3577942   0.0699044  5.1183 3.082e-07 ***
## intercept               2.6897378   2.3217003  1.1585    0.2467    
## time(dat)              -0.0016276   0.0012435 -1.3090    0.1905    
## Afterww2               23.4711228   5.7243811  4.1002 4.128e-05 ***
## IndustryRev           -25.4781150   3.6554503 -6.9699 3.172e-12 ***
## I1976                 -40.9890374   5.6387633 -7.2692 3.617e-13 ***
## time(dat):Afterww2     -0.0121119   0.0029390 -4.1211 3.770e-05 ***
## time(dat):IndustryRev   0.0133418   0.0019239  6.9347 4.072e-12 ***
## time(dat):I1976         0.0207733   0.0028617  7.2591 3.896e-13 ***
## ---
## Signif. codes:  0 '***' 0.001 '**' 0.01 '*' 0.05 '.' 0.1 ' ' 1
\end{verbatim}

\begin{Shaded}
\begin{Highlighting}[]
\CommentTok{\#model.matrix = model.matrix(object= \textasciitilde{} time(tmp)+time(tmp)*Afterww2+ time(tmp)*IndustryRev{-}1)}

\NormalTok{tmp }\OtherTok{\textless{}{-}}\NormalTok{ dat[,}\DecValTok{2}\NormalTok{][}\DecValTok{1}\SpecialCharTok{:}\DecValTok{163}\NormalTok{]}

\NormalTok{Afterww2}\FloatTok{.2}\OtherTok{\textless{}{-}}\NormalTok{Afterww2[}\DecValTok{1}\SpecialCharTok{:}\DecValTok{163}\NormalTok{]}
\NormalTok{IndustryRev}\FloatTok{.2}\OtherTok{\textless{}{-}}\NormalTok{ IndustryRev[}\DecValTok{1}\SpecialCharTok{:}\DecValTok{163}\NormalTok{]}
\NormalTok{I1976}\FloatTok{.2} \OtherTok{\textless{}{-}}\NormalTok{ I1976[}\DecValTok{1}\SpecialCharTok{:}\DecValTok{163}\NormalTok{]}
\NormalTok{model.matrix}\FloatTok{.2} \OtherTok{=} \FunctionTok{model.matrix}\NormalTok{(}\AttributeTok{object=} \SpecialCharTok{\textasciitilde{}} \FunctionTok{time}\NormalTok{(tmp)}\SpecialCharTok{+}\FunctionTok{time}\NormalTok{(tmp)}\SpecialCharTok{*}\NormalTok{Afterww2}\FloatTok{.2}\SpecialCharTok{+} \FunctionTok{time}\NormalTok{(tmp)}\SpecialCharTok{*}\NormalTok{IndustryRev}\FloatTok{.2}\SpecialCharTok{+} \FunctionTok{time}\NormalTok{(tmp)}\SpecialCharTok{*}\NormalTok{I1976}\FloatTok{.2} \SpecialCharTok{{-}}\DecValTok{1}\NormalTok{)}

\NormalTok{model.ts}\FloatTok{.2} \OtherTok{\textless{}{-}} \FunctionTok{arima}\NormalTok{(}\AttributeTok{x =}\NormalTok{ tmp, }\AttributeTok{order =} \FunctionTok{c}\NormalTok{(}\DecValTok{2}\NormalTok{,}\DecValTok{0}\NormalTok{,}\DecValTok{2}\NormalTok{),  }\AttributeTok{xreg =}\NormalTok{ model.matrix}\FloatTok{.2}\NormalTok{, }\AttributeTok{method =} \StringTok{"ML"}\NormalTok{)}

\FunctionTok{tsdiag}\NormalTok{(model.ts}\FloatTok{.2}\NormalTok{)}
\end{Highlighting}
\end{Shaded}

\includegraphics{Final-Project_files/figure-latex/unnamed-chunk-8-1.pdf}

\begin{Shaded}
\begin{Highlighting}[]
\FunctionTok{acf}\NormalTok{(}\FunctionTok{resid}\NormalTok{(model.ts}\FloatTok{.2}\NormalTok{))}
\end{Highlighting}
\end{Shaded}

\includegraphics{Final-Project_files/figure-latex/unnamed-chunk-8-2.pdf}

\begin{Shaded}
\begin{Highlighting}[]
\CommentTok{\#pacf(resid(model.ts.2))}

\CommentTok{\#coeftest(model.ts.2)}
\end{Highlighting}
\end{Shaded}

\begin{Shaded}
\begin{Highlighting}[]
\NormalTok{fitted\_values }\OtherTok{\textless{}{-}} \FunctionTok{resid}\NormalTok{(model.ts) }\SpecialCharTok{+}\NormalTok{ dat[,}\DecValTok{2}\NormalTok{]}
\FunctionTok{plot}\NormalTok{(dat[, }\DecValTok{2}\NormalTok{], }\AttributeTok{type =} \StringTok{"l"}\NormalTok{, }\AttributeTok{col =} \StringTok{"tomato"}\NormalTok{, }\AttributeTok{lwd =} \DecValTok{2}\NormalTok{, }\AttributeTok{main =} \StringTok{"Actual vs Fitted"}\NormalTok{, }\AttributeTok{ylab =} \StringTok{"Value"}\NormalTok{)}
\FunctionTok{lines}\NormalTok{(fitted\_values, }\AttributeTok{col =} \StringTok{"blue"}\NormalTok{, }\AttributeTok{lwd =} \DecValTok{2}\NormalTok{)}
\FunctionTok{legend}\NormalTok{(}\StringTok{"topright"}\NormalTok{, }\AttributeTok{legend =} \FunctionTok{c}\NormalTok{(}\StringTok{"Actual"}\NormalTok{, }\StringTok{"Fitted"}\NormalTok{), }\AttributeTok{col =} \FunctionTok{c}\NormalTok{(}\StringTok{"tomato"}\NormalTok{, }\StringTok{"blue"}\NormalTok{), }\AttributeTok{lty =} \DecValTok{1}\NormalTok{, }\AttributeTok{lwd =} \DecValTok{2}\NormalTok{)}
\end{Highlighting}
\end{Shaded}

\includegraphics{Final-Project_files/figure-latex/unnamed-chunk-9-1.pdf}

\begin{Shaded}
\begin{Highlighting}[]
\FunctionTok{plot}\NormalTok{(dat[,}\DecValTok{2}\NormalTok{])}
\FunctionTok{abline}\NormalTok{(}\AttributeTok{v=}\DecValTok{1900}\NormalTok{)}
\FunctionTok{abline}\NormalTok{(}\AttributeTok{v =} \DecValTok{1945}\NormalTok{)}
\FunctionTok{abline}\NormalTok{(}\AttributeTok{v =} \DecValTok{1976}\NormalTok{)}
\end{Highlighting}
\end{Shaded}

\includegraphics{Final-Project_files/figure-latex/unnamed-chunk-10-1.pdf}

\hypertarget{usuing-c02-to-predict}{%
\section{usuing c02 to predict}\label{usuing-c02-to-predict}}

\begin{Shaded}
\begin{Highlighting}[]
\NormalTok{prewhite }\OtherTok{\textless{}{-}} \FunctionTok{Arima}\NormalTok{(dat[,}\DecValTok{1}\NormalTok{], }\AttributeTok{model =}\NormalTok{ model.ts, }\AttributeTok{xreg =}\NormalTok{ model.matrix)}


\FunctionTok{lag2.plot}\NormalTok{(}\FunctionTok{resid}\NormalTok{(prewhite), }\FunctionTok{resid}\NormalTok{(model.ts),}\AttributeTok{max.lag =} \DecValTok{20}\NormalTok{)}
\end{Highlighting}
\end{Shaded}

\includegraphics{Final-Project_files/figure-latex/unnamed-chunk-11-1.pdf}

\begin{Shaded}
\begin{Highlighting}[]
\FunctionTok{ccf2}\NormalTok{(}\FunctionTok{resid}\NormalTok{(prewhite), }\FunctionTok{resid}\NormalTok{(model.ts))}
\end{Highlighting}
\end{Shaded}

\includegraphics{Final-Project_files/figure-latex/unnamed-chunk-11-2.pdf}

Everything from c02 is captured in our model for tempature.

\hypertarget{refs}{}
\begin{CSLReferences}{1}{0}
\leavevmode\vadjust pre{\hypertarget{ref-ritchie_2023_co}{}}%
Ritchie, Hannah, Pablo Rosado, and Max Roser. 2023. {``CO₂ and
Greenhouse Gas Emissions.''} \emph{Our World in Data}.
\url{https://ourworldindata.org/co2-and-greenhouse-gas-emissions}.

\end{CSLReferences}

\end{document}
